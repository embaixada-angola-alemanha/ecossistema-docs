% ============================================================
% Template LaTeX — Embaixada da República de Angola na Alemanha
% Ecossistema Digital — Documentação Técnica
% ============================================================
\documentclass[a4paper,11pt]{article}

% --- Pacotes ---
\usepackage[utf8]{inputenc}
\usepackage[T1]{fontenc}
\usepackage[portuguese]{babel}
\usepackage{geometry}
\usepackage{fancyhdr}
\usepackage{graphicx}
\usepackage{hyperref}
\usepackage{xcolor}
\usepackage{titlesec}
\usepackage{booktabs}
\usepackage{enumitem}
\usepackage{listings}
\usepackage{parskip}

% --- Cores institucionais (bandeira de Angola) ---
\definecolor{angola-red}{HTML}{CE1126}
\definecolor{angola-black}{HTML}{000000}
\definecolor{angola-gold}{HTML}{F9D616}

% --- Geometria da página ---
\geometry{
    top=30mm,
    bottom=25mm,
    left=25mm,
    right=25mm,
    headheight=15mm
}

% --- Cabeçalho e rodapé ---
\pagestyle{fancy}
\fancyhf{}
\fancyhead[L]{\small\textcolor{angola-red}{Embaixada da República de Angola na Alemanha}}
\fancyhead[R]{\small\textcolor{gray}{\leftmark}}
\fancyfoot[C]{\thepage}
\fancyfoot[R]{\small\textcolor{gray}{Ecossistema Digital}}
\renewcommand{\headrulewidth}{0.5pt}
\renewcommand{\footrulewidth}{0.3pt}

% --- Estilo de títulos ---
\titleformat{\section}
    {\Large\bfseries\color{angola-red}}{\thesection}{1em}{}
\titleformat{\subsection}
    {\large\bfseries\color{angola-black}}{\thesubsection}{1em}{}

% --- Hyperlinks ---
\hypersetup{
    colorlinks=true,
    linkcolor=angola-red,
    urlcolor=angola-red,
    citecolor=angola-red
}

% --- Listagens de código ---
\lstset{
    basicstyle=\ttfamily\small,
    breaklines=true,
    frame=single,
    rulecolor=\color{gray},
    backgroundcolor=\color{gray!5},
    keywordstyle=\color{angola-red},
    commentstyle=\color{gray},
    stringstyle=\color{angola-gold!80!black}
}

% ============================================================
% METADADOS DO DOCUMENTO (editar para cada documento)
% ============================================================
\newcommand{\doctitulo}{Título do Documento}
\newcommand{\docautor}{Nome do Autor}
\newcommand{\docversao}{1.0}
\newcommand{\docdata}{\today}
\newcommand{\docdepartamento}{Secção Consular}

% ============================================================
\begin{document}

% --- Página de rosto ---
\begin{titlepage}
    \centering
    \vspace*{2cm}

    {\color{angola-red}\rule{\textwidth}{2pt}}

    \vspace{1.5cm}

    {\Huge\bfseries\color{angola-red} \doctitulo \par}

    \vspace{1cm}

    {\Large Embaixada da República de Angola \par}
    {\large Berlim, Alemanha \par}

    \vspace{2cm}

    \begin{tabular}{ll}
        \textbf{Autor:}        & \docautor \\
        \textbf{Departamento:} & \docdepartamento \\
        \textbf{Versão:}       & \docversao \\
        \textbf{Data:}         & \docdata \\
    \end{tabular}

    \vfill

    {\color{angola-red}\rule{\textwidth}{2pt}}

    \vspace{0.5cm}
    {\small\textcolor{gray}{Ecossistema Digital — Documento gerado automaticamente}}
\end{titlepage}

% --- Índice ---
\tableofcontents
\newpage

% ============================================================
% CONTEÚDO (editar a partir daqui)
% ============================================================

\section{Introdução}

Este documento faz parte da documentação técnica do Ecossistema Digital da Embaixada da República de Angola na Alemanha.

\subsection{Objectivo}

Descrever o objectivo deste documento.

\subsection{Âmbito}

Definir o âmbito e os destinatários.

\section{Conteúdo Principal}

Corpo do documento.

\subsection{Subsecção}

Texto da subsecção.

\section{Conclusão}

Conclusões e próximos passos.

% ============================================================
\end{document}
