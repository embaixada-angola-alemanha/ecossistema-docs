% ============================================================
% Documentação Técnica — Ecossistema Digital
% Embaixada da República de Angola na Alemanha
% ============================================================
\documentclass[a4paper,11pt,twoside]{report}

% --- Pacotes ---
\usepackage[utf8]{inputenc}
\usepackage[T1]{fontenc}
\usepackage[portuguese]{babel}
\usepackage{geometry}
\usepackage{fancyhdr}
\usepackage{graphicx}
\usepackage{hyperref}
\usepackage{xcolor}
\usepackage{titlesec}
\usepackage{booktabs}
\usepackage{enumitem}
\usepackage{listings}
\usepackage{parskip}
\usepackage{longtable}
\usepackage{tabularx}
\usepackage{multirow}
\usepackage{float}
\usepackage{tikz}
\usepackage{pgfgantt}

% --- Cores institucionais ---
\definecolor{angola-red}{HTML}{CE1126}
\definecolor{angola-black}{HTML}{000000}
\definecolor{angola-gold}{HTML}{F9D616}
\definecolor{code-bg}{HTML}{F5F5F5}

% --- Geometria ---
\geometry{
    top=30mm, bottom=25mm, left=30mm, right=25mm, headheight=15mm
}

% --- Cabeçalho/Rodapé ---
\pagestyle{fancy}
\fancyhf{}
\fancyhead[LE]{\small\textcolor{angola-red}{Ecossistema Digital}}
\fancyhead[RO]{\small\textcolor{gray}{\rightmark}}
\fancyfoot[C]{\thepage}
\fancyfoot[RE,LO]{\small\textcolor{gray}{Embaixada de Angola — Berlim}}
\renewcommand{\headrulewidth}{0.5pt}
\renewcommand{\footrulewidth}{0.3pt}

% --- Títulos ---
\titleformat{\chapter}[display]
    {\Huge\bfseries\color{angola-red}}{\chaptertitlename\ \thechapter}{20pt}{\Huge}
\titleformat{\section}
    {\Large\bfseries\color{angola-red}}{\thesection}{1em}{}
\titleformat{\subsection}
    {\large\bfseries\color{angola-black}}{\thesubsection}{1em}{}

% --- Hyperlinks ---
\hypersetup{
    colorlinks=true, linkcolor=angola-red, urlcolor=angola-red,
    pdftitle={Ecossistema Digital - Documentação Técnica},
    pdfauthor={Embaixada de Angola na Alemanha}
}

% --- Código ---
\lstset{
    basicstyle=\ttfamily\footnotesize,
    breaklines=true, frame=single,
    rulecolor=\color{gray}, backgroundcolor=\color{code-bg},
    keywordstyle=\color{angola-red}\bfseries,
    commentstyle=\color{gray}\itshape,
    stringstyle=\color{angola-gold!80!black},
    numbers=left, numberstyle=\tiny\color{gray},
    showstringspaces=false
}

% --- Metadados ---
\newcommand{\docversao}{1.0}
\newcommand{\docdata}{Fevereiro 2026}

% ============================================================
\begin{document}

% --- Página de Rosto ---
\begin{titlepage}
    \centering
    \vspace*{1cm}
    {\color{angola-red}\rule{\textwidth}{3pt}}
    \vspace{1cm}

    {\fontsize{28}{34}\selectfont\bfseries\color{angola-red}
        Ecossistema Digital\par}
    \vspace{0.5cm}
    {\LARGE Documentação Técnica\par}
    \vspace{1cm}
    {\Large Embaixada da República de Angola\par}
    {\large Berlim, Alemanha\par}

    \vspace{2cm}

    \begin{tabular}{ll}
        \toprule
        \textbf{Versão}      & \docversao \\
        \textbf{Data}        & \docdata \\
        \textbf{Classificação} & Interno \\
        \textbf{Idioma}      & Português \\
        \bottomrule
    \end{tabular}

    \vfill
    {\color{angola-red}\rule{\textwidth}{3pt}}
    \vspace{0.3cm}
    {\small\textcolor{gray}{Documento gerado no âmbito do projecto Ecossistema Digital}}
\end{titlepage}

% --- Controlo de Versões ---
\thispagestyle{empty}
\section*{Controlo de Versões}
\begin{tabularx}{\textwidth}{llXl}
    \toprule
    \textbf{Versão} & \textbf{Data} & \textbf{Alterações} & \textbf{Autor} \\
    \midrule
    0.1 & 2026-02-16 & Estrutura inicial, infra e GPJ & Equipa Dev \\
    0.5 & 2026-02-17 & SGC, SI, WN backends e frontends & Equipa Dev \\
    0.8 & 2026-02-18 & Mobile, integração, testes & Equipa Dev \\
    1.0 & 2026-02-18 & Versão final para go-live & Equipa Dev \\
    \bottomrule
\end{tabularx}
\newpage

% --- Índice ---
\tableofcontents
\newpage

% ============================================================
\chapter{Introdução}
% ============================================================

\section{Objectivo}

Este documento constitui a documentação técnica completa do Ecossistema Digital da Embaixada da República de Angola na Alemanha. Destina-se à equipa técnica responsável pela manutenção, evolução e operação dos sistemas.

\section{Âmbito}

O Ecossistema Digital é composto por quatro sistemas principais, uma aplicação móvel e uma camada de infraestrutura partilhada:

\begin{enumerate}
    \item \textbf{SGC} — Sistema de Gestão Consular (gestão de cidadãos, vistos, agendamentos, documentos)
    \item \textbf{SI} — Site Institucional (website público da embaixada)
    \item \textbf{WN} — Web Notícias (portal de notícias)
    \item \textbf{GPJ} — Gestão de Projectos (gestão interna de sprints e tarefas)
    \item \textbf{Mobile} — Aplicação React Native (iOS + Android para cidadãos)
    \item \textbf{Infra} — Infraestrutura partilhada (Kubernetes, CI/CD, monitoring)
\end{enumerate}

\section{Convenções}

\begin{itemize}
    \item Código-fonte: \texttt{monospace}
    \item URLs e endpoints: \texttt{/api/v1/recurso}
    \item Variáveis de ambiente: \texttt{VARIAVEL\_NOME}
    \item Referências a ficheiros: \textit{itálico}
\end{itemize}

\section{Arquitectura Geral}

O ecossistema segue uma arquitectura \textbf{multi-repositório} com backend em Java 21/Spring Boot 3.4, frontends administrativos em Angular 18, e aplicação móvel em React Native 0.76. A autenticação é centralizada via Keycloak com OAuth2/OIDC.

% ============================================================
\chapter{Stack Tecnológico}
% ============================================================

\section{Backend}

\begin{tabularx}{\textwidth}{lXl}
    \toprule
    \textbf{Componente} & \textbf{Tecnologia} & \textbf{Versão} \\
    \midrule
    Runtime & Java (Temurin) & 21 LTS \\
    Framework & Spring Boot & 3.4.x \\
    Segurança & Spring Security + OAuth2 & 6.x \\
    Base de dados & PostgreSQL & 16 \\
    Migrações & Flyway & 10.x \\
    Cache & Redis & 7.x \\
    Mensagens & RabbitMQ & 3.13 \\
    Ficheiros & MinIO (S3-compatible) & latest \\
    Build & Maven & 3.9.x \\
    \bottomrule
\end{tabularx}

\section{Frontend (Admin)}

\begin{tabularx}{\textwidth}{lXl}
    \toprule
    \textbf{Componente} & \textbf{Tecnologia} & \textbf{Versão} \\
    \midrule
    Framework & Angular & 18.x \\
    UI Library & Angular Material & 18.x \\
    State Management & NgRx (onde aplicável) & 18.x \\
    Charts & Chart.js + ng2-charts & 4.x \\
    i18n & @ngx-translate/core & 16.x \\
    Build & Angular CLI & 18.x \\
    \bottomrule
\end{tabularx}

\section{Mobile}

\begin{tabularx}{\textwidth}{lXl}
    \toprule
    \textbf{Componente} & \textbf{Tecnologia} & \textbf{Versão} \\
    \midrule
    Framework & React Native & 0.76.x \\
    Navegação & React Navigation & 7.x \\
    Data Fetching & TanStack React Query & 5.x \\
    State & Zustand & 5.x \\
    HTTP & Axios & 1.7.x \\
    i18n & i18next + react-i18next & 24.x / 15.x \\
    Push & Firebase Cloud Messaging & latest \\
    OTA Updates & CodePush & 9.x \\
    E2E Tests & Detox & 20.x \\
    \bottomrule
\end{tabularx}

\section{Infraestrutura}

\begin{tabularx}{\textwidth}{lXl}
    \toprule
    \textbf{Componente} & \textbf{Tecnologia} & \textbf{Versão} \\
    \midrule
    Containers & Docker & 25.x \\
    Orquestração & Kubernetes (K3s) & 1.30.x \\
    CI/CD & GitHub Actions + ArgoCD & latest \\
    Autenticação & Keycloak & 24.x \\
    Monitoring & Prometheus + Grafana & latest \\
    Logs & Loki + Promtail & latest \\
    Ingress & Traefik & 3.x \\
    DNS/TLS & Let's Encrypt + cert-manager & latest \\
    \bottomrule
\end{tabularx}

% ============================================================
\chapter{Arquitectura dos Sistemas}
% ============================================================

\section{Visão Geral}

Cada sistema segue o padrão \textbf{Backend API + Frontend SPA}, comunicando via REST APIs autenticadas por JWT tokens emitidos pelo Keycloak.

\subsection{Comunicação entre Sistemas}

\begin{itemize}
    \item \textbf{SGC $\leftrightarrow$ SI}: Eventos publicados via RabbitMQ (novo evento consular $\rightarrow$ página SI)
    \item \textbf{SGC $\leftrightarrow$ WN}: Notificações de serviço $\rightarrow$ artigos WN
    \item \textbf{SGC $\leftrightarrow$ Mobile}: API REST directa, push via FCM
    \item \textbf{GPJ}: Monitoriza todos os sistemas via health checks
\end{itemize}

\section{SGC — Sistema de Gestão Consular}

\subsection{Domínio}

O SGC gere todo o fluxo consular:

\begin{itemize}
    \item \textbf{Cidadão}: Registo, pesquisa, actualização, desactivação
    \item \textbf{Visto}: 7 tipos (TURISMO, TRABALHO, ESTUDO, NEGOCIO, TRANSITO, RESIDENCIA, DIPLOMATICO), 9 estados
    \item \textbf{Agendamento}: 7 tipos, 6 estados, detecção de conflitos, QR check-in
    \item \textbf{Documento}: Upload/download para MinIO, associado a processo
    \item \textbf{Notificação}: Sistema interno + push para mobile
\end{itemize}

\subsection{Portos}

\begin{tabular}{ll}
    \toprule
    Backend API & \texttt{:8081} \\
    Frontend & \texttt{:3001} \\
    \bottomrule
\end{tabular}

\subsection{Roles RBAC}

\begin{tabular}{lp{8cm}}
    \toprule
    \textbf{Role} & \textbf{Permissões} \\
    \midrule
    ADMIN & Acesso total: CRUD cidadãos, vistos, agendamentos, relatórios, configurações \\
    OFFICER & CRUD cidadãos, vistos, agendamentos (sem configurações) \\
    VIEWER & Leitura apenas \\
    \bottomrule
\end{tabular}

\section{SI — Site Institucional}

\subsection{Domínio}

Website público multilingue (PT/EN/DE) com:

\begin{itemize}
    \item Páginas estáticas (Embaixador, Serviços, Contactos)
    \item Eventos (criação, publicação, arquivo)
    \item Conteúdo gerido por CMS backend
\end{itemize}

\subsection{Portos}

\begin{tabular}{ll}
    \toprule
    Backend API & \texttt{:8082} \\
    Frontend & \texttt{:3002} \\
    \bottomrule
\end{tabular}

\section{WN — Web Notícias}

\subsection{Domínio}

Portal de notícias com:

\begin{itemize}
    \item Artigos (DRAFT $\rightarrow$ PUBLISHED), categorias, tags
    \item Pesquisa full-text (PostgreSQL tsvector)
    \item Feed RSS, partilha social
    \item Media library (imagens optimizadas)
\end{itemize}

\subsection{Portos}

\begin{tabular}{ll}
    \toprule
    Backend API & \texttt{:8083} \\
    Frontend & \texttt{:3003} \\
    \bottomrule
\end{tabular}

\section{GPJ — Gestão de Projectos}

\subsection{Domínio}

Sistema interno de gestão de projectos:

\begin{itemize}
    \item Sprints, tarefas (Kanban board), time tracking
    \item Gráfico de Gantt, velocity chart, burn-down
    \item Monitorização de saúde de todos os sistemas
\end{itemize}

\subsection{Portos}

\begin{tabular}{ll}
    \toprule
    Backend API & \texttt{:8084} \\
    Frontend & \texttt{:4200} \\
    \bottomrule
\end{tabular}

\section{Mobile}

\subsection{Funcionalidades}

\begin{itemize}
    \item Autenticação Keycloak via WebView
    \item Perfil do cidadão + upload de documentos
    \item Solicitação e tracking de vistos
    \item Agendamento com QR check-in
    \item Feed de notícias (WN)
    \item Push notifications (FCM)
    \item Modo offline com sync automático
    \item 4 idiomas: PT, EN, DE, FR
    \item OTA updates via CodePush
\end{itemize}

% ============================================================
\chapter{Base de Dados}
% ============================================================

\section{Modelo de Dados — SGC}

\subsection{Tabelas Principais}

\begin{longtable}{llp{6cm}}
    \toprule
    \textbf{Tabela} & \textbf{Colunas-chave} & \textbf{Descrição} \\
    \midrule
    \endhead
    cidadao & id, nome, email, numero\_passaporte, nif, nacionalidade & Cidadão registado \\
    visto & id, cidadao\_id, tipo, estado, motivo, data\_entrada, data\_saida & Pedido de visto \\
    agendamento & id, cidadao\_id, tipo, estado, data\_hora, notas & Marcação consular \\
    processo & id, cidadao\_id, tipo, estado & Processo consular \\
    documento & id, processo\_id, nome, path, tipo\_mime, tamanho & Documento carregado \\
    notificacao & id, cidadao\_id, titulo, mensagem, lida & Notificação interna \\
    \bottomrule
\end{longtable}

\subsection{Índices Recomendados}

\begin{lstlisting}[language=SQL]
CREATE INDEX idx_cidadao_nome ON cidadao (lower(nome));
CREATE INDEX idx_cidadao_passaporte ON cidadao (numero_passaporte);
CREATE INDEX idx_processo_cidadao_estado ON processo (cidadao_id, estado);
CREATE INDEX idx_visto_cidadao_estado ON visto (cidadao_id, estado);
CREATE INDEX idx_agendamento_datahora ON agendamento (data_hora);
CREATE INDEX idx_agendamento_conflict ON agendamento (data_hora, tipo, estado)
    WHERE estado IN ('PENDENTE', 'CONFIRMADO');
\end{lstlisting}

\section{Modelo de Dados — WN}

\begin{longtable}{llp{6cm}}
    \toprule
    \textbf{Tabela} & \textbf{Colunas-chave} & \textbf{Descrição} \\
    \midrule
    article & id, slug, titulo, resumo, conteudo, status, published\_at & Artigo de notícia \\
    category & id, nome, slug & Categoria \\
    tag & id, nome, slug & Etiqueta \\
    article\_category & article\_id, category\_id & Relação N:M \\
    media & id, filename, path, tipo\_mime, tamanho & Ficheiro multimédia \\
    \bottomrule
\end{longtable}

\section{Modelo de Dados — SI}

\begin{longtable}{llp{6cm}}
    \toprule
    \textbf{Tabela} & \textbf{Colunas-chave} & \textbf{Descrição} \\
    \midrule
    page & id, slug, titulo, conteudo, status & Página do site \\
    event & id, titulo, descricao, start\_date, end\_date, location, status & Evento \\
    \bottomrule
\end{longtable}

\section{Modelo de Dados — GPJ}

\begin{longtable}{llp{6cm}}
    \toprule
    \textbf{Tabela} & \textbf{Colunas-chave} & \textbf{Descrição} \\
    \midrule
    sprint & id, titulo, data\_inicio, data\_fim, estado & Sprint \\
    task & id, sprint\_id, nome, descricao, estado, assignee\_id, horas & Tarefa \\
    time\_log & id, task\_id, horas, descricao, logged\_date & Registo de tempo \\
    \bottomrule
\end{longtable}

\section{Migrações (Flyway)}

Cada backend mantém migrações em \texttt{src/main/resources/db/migration/} com a convenção:

\begin{lstlisting}
V{versao}__{descricao}.sql
Exemplo: V001__create_cidadao_table.sql
\end{lstlisting}

% ============================================================
\chapter{API REST}
% ============================================================

\section{Convenções}

\begin{itemize}
    \item Base path: \texttt{/api/v1/}
    \item Autenticação: Bearer JWT token (Keycloak)
    \item Formato: JSON (UTF-8)
    \item Paginação: \texttt{?page=0\&size=20\&sort=nome,asc}
    \item Erros: \texttt{\{error, message, status, timestamp\}}
\end{itemize}

\section{SGC API — Endpoints Principais}

\begin{longtable}{llp{5cm}}
    \toprule
    \textbf{Método} & \textbf{Endpoint} & \textbf{Descrição} \\
    \midrule
    \endhead
    GET & /api/v1/cidadaos & Listar cidadãos (paginado) \\
    POST & /api/v1/cidadaos & Registar cidadão \\
    GET & /api/v1/cidadaos/\{id\} & Obter cidadão \\
    PUT & /api/v1/cidadaos/\{id\} & Actualizar cidadão \\
    \midrule
    GET & /api/v1/vistos & Listar vistos \\
    POST & /api/v1/vistos & Criar pedido de visto \\
    GET & /api/v1/vistos/\{id\} & Obter visto \\
    PATCH & /api/v1/vistos/\{id\}/estado & Mudar estado do visto \\
    GET & /api/v1/vistos/\{id\}/timeline & Timeline do visto \\
    \midrule
    GET & /api/v1/agendamentos & Listar agendamentos \\
    POST & /api/v1/agendamentos & Criar agendamento \\
    GET & /api/v1/agendamentos/slots & Slots disponíveis \\
    PATCH & /api/v1/agendamentos/\{id\}/cancelar & Cancelar agendamento \\
    \midrule
    POST & /api/v1/documentos & Upload documento \\
    GET & /api/v1/documentos/\{id\}/download & Download documento \\
    \midrule
    GET & /api/v1/actuator/health & Health check \\
    \bottomrule
\end{longtable}

\section{SI API — Endpoints Principais}

\begin{longtable}{llp{5cm}}
    \toprule
    \textbf{Método} & \textbf{Endpoint} & \textbf{Descrição} \\
    \midrule
    GET & /api/v1/public/pages & Listar páginas públicas \\
    GET & /api/v1/public/pages/\{slug\} & Obter página por slug \\
    GET & /api/v1/public/events & Listar eventos \\
    GET & /api/v1/public/events/\{id\} & Obter evento \\
    POST & /api/v1/admin/pages & Criar página (admin) \\
    PUT & /api/v1/admin/pages/\{id\} & Editar página (admin) \\
    POST & /api/v1/admin/events & Criar evento (admin) \\
    \bottomrule
\end{longtable}

\section{WN API — Endpoints Principais}

\begin{longtable}{llp{5cm}}
    \toprule
    \textbf{Método} & \textbf{Endpoint} & \textbf{Descrição} \\
    \midrule
    GET & /api/v1/public/articles & Listar artigos publicados \\
    GET & /api/v1/public/articles/\{slug\} & Obter artigo por slug \\
    GET & /api/v1/public/categories & Listar categorias \\
    GET & /api/v1/public/search?q= & Pesquisa full-text \\
    POST & /api/v1/admin/articles & Criar artigo (admin) \\
    PUT & /api/v1/admin/articles/\{id\} & Editar artigo (admin) \\
    PATCH & /api/v1/admin/articles/\{id\}/publish & Publicar artigo \\
    \bottomrule
\end{longtable}

\section{GPJ API — Endpoints Principais}

\begin{longtable}{llp{5cm}}
    \toprule
    \textbf{Método} & \textbf{Endpoint} & \textbf{Descrição} \\
    \midrule
    GET & /api/v1/sprints & Listar sprints \\
    POST & /api/v1/sprints & Criar sprint \\
    GET & /api/v1/sprints/\{id\}/tasks & Tarefas do sprint \\
    POST & /api/v1/tasks & Criar tarefa \\
    PATCH & /api/v1/tasks/\{id\}/status & Mudar estado da tarefa \\
    POST & /api/v1/timelogs & Registar tempo \\
    GET & /api/v1/dashboard & Dashboard KPIs \\
    \bottomrule
\end{longtable}

% ============================================================
\chapter{Segurança}
% ============================================================

\section{Autenticação (Keycloak)}

\begin{itemize}
    \item Realm: \texttt{ecossistema}
    \item Clients: \texttt{sgc-app}, \texttt{gpj-app}, \texttt{mobile-app}, \texttt{public-web}
    \item Fluxo: Authorization Code + PKCE
    \item Token lifetime: Access 5 min, Refresh 30 min
    \item MFA: Activado para roles ADMIN
\end{itemize}

\section{RBAC}

Roles são definidos no Keycloak e propagados via JWT claims:

\begin{lstlisting}
{
  "realm_access": {
    "roles": ["sgc-admin", "gpj-viewer"]
  }
}
\end{lstlisting}

\section{Segurança de Dados}

\begin{itemize}
    \item HTTPS obrigatório (TLS 1.3)
    \item Dados sensíveis encriptados em repouso (PostgreSQL TDE)
    \item Passwords nunca armazenadas (delegadas ao Keycloak)
    \item File uploads validados (tipo MIME, tamanho máximo 10MB)
    \item CORS restrito aos domínios do ecossistema
    \item Rate limiting: 100 req/min por IP
    \item Security headers: CSP, X-Frame-Options, X-Content-Type-Options
\end{itemize}

\section{GDPR}

\begin{itemize}
    \item Base legal: Consentimento + Interesse legítimo (Art. 6)
    \item Dados pessoais inventariados por sistema
    \item Direito de acesso e exportação implementado
    \item Direito ao esquecimento: soft-delete com anonimização após 30 dias
    \item Retenção: dados consulares 10 anos (obrigação legal)
    \item DPO designado, registos de processamento mantidos
\end{itemize}

% ============================================================
\chapter{Infraestrutura}
% ============================================================

\section{Kubernetes}

Cluster K3s com os seguintes namespaces:

\begin{tabular}{lp{8cm}}
    \toprule
    \textbf{Namespace} & \textbf{Componentes} \\
    \midrule
    ecossistema-prod & SGC, SI, WN, GPJ (backends + frontends) \\
    ecossistema-staging & Réplica para staging/UAT \\
    ecossistema-infra & PostgreSQL, Redis, RabbitMQ, MinIO \\
    keycloak & Keycloak + PostgreSQL \\
    monitoring & Prometheus, Grafana, Loki, Promtail \\
    ingress & Traefik, cert-manager \\
    \bottomrule
\end{tabular}

\section{CI/CD Pipeline}

\begin{enumerate}
    \item Commit em branch feature $\rightarrow$ GitHub Actions executa testes
    \item Merge em main $\rightarrow$ Build Docker image, push para registry
    \item ArgoCD detecta nova imagem $\rightarrow$ Deploy automático em staging
    \item Aprovação manual $\rightarrow$ Deploy em produção
\end{enumerate}

\section{Monitoring}

\begin{itemize}
    \item \textbf{Prometheus}: Métricas JVM, HTTP, PostgreSQL, Redis
    \item \textbf{Grafana}: Dashboards por sistema (4 dashboards)
    \item \textbf{Loki}: Agregação de logs centralizada
    \item \textbf{Alertas}: Slack webhook para CPU >80\%, memória >85\%, erros 5xx >1\%
\end{itemize}

% ============================================================
\chapter{Testes}
% ============================================================

\section{Estratégia de Testes}

\begin{tabular}{llp{5cm}l}
    \toprule
    \textbf{Nível} & \textbf{Ferramenta} & \textbf{Cobertura} & \textbf{CI} \\
    \midrule
    Unit & JUnit 5 / Jasmine & Services, controllers & Cada push \\
    Integration & Spring Boot Test & API endpoints & Cada push \\
    E2E (Web) & Cypress 15 & 21 test files, 4 sistemas & Merge em main \\
    E2E (Mobile) & Detox 20 & 6 test suites, 25+ cenários & Merge em main \\
    Load & JMeter 5.6 & 100 users/sistema, p95 <3s & Semanal \\
    Security & OWASP ZAP + Trivy & APIs, deps, IaC & Semanal \\
    Performance & Lighthouse CI & 10 URLs, score >80\% & Semanal \\
    UAT & Manual (55 cenários) & 320+ passos de teste & Pre-go-live \\
    \bottomrule
\end{tabular}

% ============================================================
\chapter{Plano de Projecto}
% ============================================================

\section{Sprints}

\begin{longtable}{clp{5cm}rl}
    \toprule
    \textbf{Sprint} & \textbf{Fase} & \textbf{Título} & \textbf{Horas} & \textbf{Estado} \\
    \midrule
    \endhead
    S0 & Infra & Infraestrutura \& Fundação & 40h & CONCLUIDA \\
    S1 & GPJ & GPJ Sistema Completo & 48h & CONCLUIDA \\
    S2-S5 & SGC-BE & SGC Backend (4 sprints) & 160h & CONCLUIDA \\
    S6-S9 & SGC-FE & SGC Frontend (4 sprints) & 160h & CONCLUIDA \\
    S10-S11 & SI & Site Institucional & 80h & CONCLUIDA \\
    S12-S13 & WN & Web Notícias & 80h & CONCLUIDA \\
    S14 & Integração & Integração \& DevOps & 50h & CONCLUIDA \\
    S15-S16 & Mobile & Mobile App (2 sprints) & 80h & CONCLUIDA \\
    S17 & Testes & Testes \& Segurança & 40h & CONCLUIDA \\
    S18 & Testes & UAT \& Documentação & 40h & EM CURSO \\
    S19 & Go-Live & Go-Live \& Formação & 40h & PENDENTE \\
    \bottomrule
\end{longtable}

\section{Métricas do Projecto}

\begin{tabular}{lr}
    \toprule
    Total de sprints & 20 \\
    Total de tarefas & 72 \\
    Horas planeadas & 858h \\
    Horas consumidas & 808h \\
    Progresso & 84.7\% \\
    \bottomrule
\end{tabular}

% ============================================================
\chapter{Glossário}
% ============================================================

\begin{longtable}{lp{10cm}}
    \toprule
    \textbf{Termo} & \textbf{Definição} \\
    \midrule
    \endhead
    API & Application Programming Interface \\
    CI/CD & Continuous Integration / Continuous Delivery \\
    CORS & Cross-Origin Resource Sharing \\
    CRUD & Create, Read, Update, Delete \\
    FCM & Firebase Cloud Messaging \\
    GDPR & General Data Protection Regulation \\
    JWT & JSON Web Token \\
    K3s & Lightweight Kubernetes distribution \\
    OIDC & OpenID Connect \\
    OTA & Over-The-Air (updates) \\
    PKCE & Proof Key for Code Exchange \\
    RBAC & Role-Based Access Control \\
    REST & Representational State Transfer \\
    SGC & Sistema de Gestão Consular \\
    SI & Site Institucional \\
    SPA & Single Page Application \\
    SSO & Single Sign-On \\
    SUS & System Usability Scale \\
    UAT & User Acceptance Testing \\
    WN & Web Notícias \\
    \bottomrule
\end{longtable}

% ============================================================
\end{document}
